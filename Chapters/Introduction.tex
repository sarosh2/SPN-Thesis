% Chapter 1

\chapter{Introduction} % Main chapter title

\label{Introduction} % For referencing the chapter elsewhere, use \ref{Chapter1} 

%----------------------------------------------------------------------------------------

% Define some commands to keep the formatting separated from the content 
\newcommand{\keyword}[1]{\textbf{#1}}
\newcommand{\tabhead}[1]{\textbf{#1}}
\newcommand{\code}[1]{\texttt{#1}}
\newcommand{\file}[1]{\texttt{\bfseries#1}}
\newcommand{\option}[1]{\texttt{\itshape#1}}

%----------------------------------------------------------------------------------------

The Perceptron, introduced by Frank Rosenblatt in 1958, is one of the earliest and most influential models in machine learning (ML). Designed to solve binary classification problems, the perceptron is a simple linear classifier inspired by the biological neurons in the human brain. It works by assigning weights to a set of input features and calculating a weighted sum of these features, which is passed through an activation function to determine the classification. The model iteratively adjusts these weights based on prediction errors, learning from the data until it reaches an optimal configuration.

However, while the perceptron performs well for linearly separable data, it struggles with more complex, non-linear problems. This limitation prompted the development of the Multi-Layer Perceptron (MLP) in the 1980s. MLPs extend the perceptron by adding multiple layers of perceptrons, enabling the model to learn non-linear relationships within the data. The breakthrough of backpropagation, introduced by Paul Werbos in the 1970s and popularized by Geoffrey Hinton and others in the 1980s, made it possible to train MLPs by calculating gradients and adjusting the weights across multiple layers.

The evolution of neural networks has given rise to specialized architectures such as Convolutional Neural Networks (CNNs) for image tasks and Transformer-based Large Language Models (LLMs) for natural language processing. Despite this, the MLP remains a foundational architecture that laid the groundwork for more advanced models. MLPs consist of multiple layers where each neuron in one layer is fully connected to every neuron in the subsequent layer, enabling the model to learn complex patterns in data, such as those found in image and speech recognition.

Recent research in neural network architecture design has evolved into a field known as Neural Architecture Search (NAS), which aims to find optimal architectures for a specific task by exploring various combinations of hyperparameters, layers, and connections. NAS employs search strategies, such as reinforcement learning, evolutionary algorithms, and gradient-based methods, to efficiently navigate the vast space of possible architectures. Despite the progress made, traditional MLPs are constrained by hierarchical connections between layers, which limit their ability to explore more flexible connectivity patterns, such as those involving skip connections within or between layers. While advancements like Residual Networks (ResNet) have introduced skip connections to alleviate this issue, the full potential of neuron connectivity remains untapped.

This thesis proposes a novel approach, Sarosh's Perceptron Networks (SPNs), to address these limitations by enabling unrestricted connectivity between neurons. Unlike traditional MLPs, SPNs eliminate the constraints on layer-to-layer connections, allowing neurons to interact more freely across the network. By fostering greater flexibility in connectivity, SPNs are hypothesized to enhance model performance without introducing the computational overhead typically associated with large, fully connected networks. This work aims to improve the expressiveness and efficiency of neural network architectures, offering new avenues for both theoretical exploration and practical applications in machine learning.


\section{Background and Motivation}
The Perceptron marked the first significant step in the evolution of artificial neural networks (ANNs). Rosenblatt’s model was inspired by the brain's structure, where neurons are connected by synapses and can adjust their strengths (synaptic weights) based on learning experiences. The Perceptron algorithm iterates over a set of input data, adjusts weights, and improves its ability to classify data. However, the perceptron could only solve problems that were linearly separable, leading to the need for more sophisticated architectures.

In the early 1980s, the Multi-Layer Perceptron (MLP) emerged, and the backpropagation algorithm was introduced to train these networks. Backpropagation allowed the MLPs to learn complex, non-linear functions by adjusting the weights across multiple layers using gradient descent. This innovation enabled the training of deep networks and formed the basis for modern deep learning techniques.

While MLPs were revolutionary, challenges remained regarding the optimal configuration of these networks. The number of layers and neurons in each layer, as well as how to connect these neurons, remained an open question. While the basic idea was that deeper networks (with more layers) could learn more complex patterns, no clear guidelines existed for how deep or wide the networks should be. Researchers began exploring empirical approaches to determine the ideal architecture.

One of the key insights came from "Efficient Backprop" (1998) by Yann LeCun, Léon Bottou, Geneviève Orr, and Klaus-Robert Müller, which highlighted methods for optimizing backpropagation. They proposed better weight initialization, momentum, and adaptive learning rates to accelerate convergence and prevent models from getting stuck in local minima. These methods made it possible to train deeper networks more effectively.

Further breakthroughs in architecture design came with Deep Residual Networks (ResNets), introduced in 2015 by Kaiming He and colleagues. ResNets utilized skip connections (residual blocks) that allowed information to bypass intermediate layers. This innovation solved the vanishing gradient problem, enabling the training of networks with hundreds or thousands of layers.

However, despite these advancements, the architectural design of neural networks remains an empirical challenge. Neural Architecture Search (NAS) emerged as an automated way to discover optimal architectures, but it is still computationally expensive and often depends on pre-existing knowledge.
%----------------------------------------------------------------------------------------

\section{Problem Statement}

Although MLPs and other deep learning models have made significant strides in various tasks, the design of their architecture remains a critical issue. Traditional MLPs suffer from a limited connectivity between neurons. Neurons within a single layer do not interact with one another, and neurons in different layers are connected only via intermediate neurons. This design limits the network's ability to learn complex, interdependent features in the data.

Despite advancements such as skip connections and residual networks, these solutions do not fully address the limitations of traditional MLP architectures. There is still a gap in the literature regarding the optimal arrangement of weights and neuron connectivity, which could potentially improve model performance, reduce training time, and allow for smaller, more efficient networks.
%----------------------------------------------------------------------------------------

\section{Research Gap and Contributions}

There is a significant gap in the literature concerning the optimization of neural network architectures, particularly in terms of neuron connectivity. Traditional MLPs rely on fixed architectural structures that restrict how neurons in different layers interact. Although skip connections have improved deep networks by mitigating the vanishing gradient problem, they still impose certain constraints, such as requiring neurons to have the same dimensionality or limiting connections to adjacent layers.

This thesis introduces Denser Perceptron Networks (DPNs), a new framework that eliminates the restriction of fixed connectivity patterns between neurons. DPNs allow for fully connected networks, where any two neurons can be linked. This increased connectivity has the potential to improve the model’s ability to learn complex, interdependent features, potentially leading to improved generalization and performance.

The main contributions of this thesis are:

The introduction of DPNs: A new neural network architecture that eliminates the restrictions of traditional MLP designs.

Empirical evaluation: A comparison of DPNs with traditional MLPs to determine whether the increased connectivity results in better performance, faster training, and more efficient models.

Improved training efficiency: Investigation into whether DPNs can achieve high performance while reducing the time and memory costs typically associated with larger, deeper networks.

%----------------------------------------------------------------------------------------

\section{Research Questions}

This thesis seeks to answer the following research questions:

Can Denser Perceptron Networks (DPNs) achieve improved model performance compared to traditional MLPs?

Does the removal of connectivity restrictions in neural networks improve the learning capabilities of the model?

Can DPNs maintain performance with smaller network sizes, thus improving computational efficiency?

How do DPNs compare to traditional architectures in terms of training time, memory consumption, and model accuracy?

How does the flexibility of DPNs affect their ability to generalize across different types of datasets and tasks?

%----------------------------------------------------------------------------------------

\section{Scope and Limitations}

This research focuses on Denser Perceptron Networks (DPNs) and evaluates their potential to enhance neural network performance through improved connectivity. The scope of the thesis is limited to the empirical comparison of DPNs with traditional MLPs on classification tasks.

Key limitations include:

Computational Constraints: DPNs, by design, involve a large number of connections, which may result in higher computational demands for training and inference. This research will focus on evaluating whether the performance gains from DPNs justify the computational cost.

Task-specific Design: The results of this study may vary depending on the dataset and task. While the thesis focuses on classification tasks, the generalizability of DPNs to other types of tasks may require further investigation.

Data Constraints: The experiments will be conducted using a set of standard datasets, and the findings may not fully apply to more complex or diverse real-world data.

%----------------------------------------------------------------------------------------


\section{Thesis Structure and Overview}

The remainder of this thesis is organized as follows:
\begin{enumerate}
    \item Chapter \ref{LiteratureReview} presents a literature review on the existing research on continual learning and the different methods proposed for the mitigation of catastrophic forgetting.
    \item Chapter \ref{Methodology} describes the methodology used in the research, including the method for data preparation, preprocessing, fine-tuning, implementation of the mitigation approach, and the evaluation with different benchmarks.
    \item Chapter \ref{Experiments} describes the different experiments carried out as part of this study, along with the different ablations performed.
    \item Chapter \ref{Results} involves the analysis of the results obtained from the experiments.
    \item Chapter \ref{Conclusion} summarizes the findings of the thesis work and suggests directions for future work.

\end{enumerate}